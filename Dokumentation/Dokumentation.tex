%
% File Name     : Protokoll.tex
% Purpose       :
% Creation Date : 20-12-2013
% Last Modified : Sat 11 Jan 2014 10:37:59 PM CET
% Created By    :
%
\documentclass[a4paper,12pt]{article}
\usepackage{amsfonts}
\usepackage{amssymb}
\usepackage{listings}
\usepackage{tabularx}
\usepackage[utf8]{inputenc}
\usepackage[ngerman]{babel}
\usepackage{fancyhdr}
\usepackage{alltt}
\usepackage{amsmath}
\usepackage{graphicx}
\usepackage[hidelinks]{hyperref}
\usepackage{biblatex}
\pagestyle{fancy}
\fancyhead[R]{\today}
\fancyhead[L]{Projekt 2}

\begin{document}
\title{Projekt 2 - Gruppe 8}
\author{ Oliver Gebhard \and Noel Kuntze \and
Andrey Nikolaev \and Anna Ostrovskaya }
\maketitle


\newpage
\tableofcontents

\newpage
\setcounter{tocdepth}{2}

\newpage

\section{Einleitung}

Im Zuge dieser Projektarbeit soll ein Known-Plaintext-Angriff auf eine einfache Stromchiffre realisiert werden.
Und im zweiten Schritt geknackt werden.
Hierbei soll es möglich sein mit einem gegebenen Klartext/ Chiffretext-Paar den Schlüssel (oder einen Schlüssel), der zum verschlüsseln des Klartextes verwendet wurde, berechnet werden.
Nachfolgend wird der Klartext mit m (Message) der Chiffretext mit c (Cipher) und der Schlüssel je nach Kontext mit k (Key) oder a (lt. Aufgabenstellung) abgekürzt.
Das Ergebnis aus Klartext XOR Chiffretext wird mit z abgekürzt (lt. Aufgabenstellung).

\subsection{Szenario}

Gegeben ist die Funktionsweise eines Pseudozufallszahlengenerator mit dem Namen Crappy-One, der die Basis für eine Stromchiffre darstellt.
Diese Stromchiffre wird mit einem 10 Byte langen Schlüssel gefüttert und verschlüsselt damit ein 16 Byte langen Klartext.
Aus dem Klartext und dem resultierenden Chiffretext soll nun der Schlüssel wiederhergestellt werden.
Der Wiederherstellungsprozess des Schlüssels soll über einen Backtracking-Algorithmus realisiert werden.

\subsection{Problemstellung}

Das Problem besteht im Groben aus zwei Komponenten: 
\begin{itemize}
\item den Zufallszahlengenerator, um den die Stromchiffre gebaut wird
\item dem Backtracking-Algorithmus, zum finden eines passenden Schlüssels
\end{itemize}

\section{Umsetzung}

\begin{alltt}
Legende:
a - Schlüsselbyte
z - Verschlüsselungsbyte
\( \oplus \) steht für XOR mit Rotation um 5 Bit nach rechts.
\( \boxplus \) steht für Addition MOD 256.
Um die Verwirrung in Grenzen zu halten wird ab 0 beginnend gezählt.

Allgemein:

Wobei i die aktuelle Runde bzw. der aktuelle Takt darstellt.
Für die Runden 0 bis 6 gilt für z (das Verschlüsselungsbyte):

\( z\sb{i} \leftarrow a\sb{i} \boxplus a\sb{i+3} \)

Ab Runde 7 bis 9 wird das erste mal ein bereits überschriebenes 
Schlüsselbyte verwendet:

\( z\sb{i} \leftarrow a\sb{i \mod 10} \boxplus (a\sb{i+3 \mod 10} \oplus a\sb{i+8 \mod 10}) \)

Von Runde 10 bis 14 werden zwei Schlüsselbytes die schon überschrieben wurden benutzt.
In Runde 15 wurde ein Schlüsselbyte bereits das zweite mal überschrieben.
\footnote{Anhang}
\end{alltt}

\subsection{Zufallszahlengenerator}

Schlüsselstromgenerator
Linearesrückkopplungsschieberegister

Beschreibung

\begin{verbatim}

\end{verbatim}

\footnote{http://en.wikipedia.org/LFSR}

\subsection{Backtracking}

Allgemeine Herangehensweise: \\
Pseudocode

\begin{alltt}
Backtrack(input) {

    //Prüfe Teillösung
    IF (Teillösung falsch) RETURN

    //Prüfe Gesamtlösung
    IF (Lösung korrekt) 
        OUTPUT Lösung
        RETURN
    ELSE 
        RETURN

    //Erweitere Teillösung
    FOR ALL
        Backtrack(input + 1)
}
\end{alltt}


\section{Analyse}

\subsection{Vernam-Chiffre}

$c_i = m_i \oplus z_i$

\subsection{One-Time-Pad}

\subsection{Vollständige Suche}

Probiere jede mögliche Kombination


\newpage
\begin{appendix}

\section{Anhang}
\subsection{LFSR Runden}

\begin{alltt}
Was passiert in jeder Runde?
Kein Anspruch auf Richtigkeit:

Innerer Zustand in Runde 0:
\( a\sb{0} | a\sb{1} | a\sb{2} | a\sb{3} | a\sb{4} | a\sb{5} | a\sb{6} | a\sb{7} | a\sb{8} | a\sb{9} \)

\( z\sb{0} \leftarrow a\sb{0} \boxplus a\sb{3}\)
\( a'\sb{0} \leftarrow a\sb{0} \oplus a\sb{5} \)

Innerer Zustand in Runde 1:
\( a'\sb{0} | a\sb{1} | a\sb{2} | a\sb{3} | a\sb{4} | a\sb{5} | a\sb{6} | a\sb{7} | a\sb{8} | a\sb{9} \)

\( z\sb{1} \leftarrow a\sb{1} \boxplus a\sb{4}\)
\( a'\sb{1} \leftarrow a\sb{1} \oplus a\sb{6} \)

Innerer Zustand in Runde 2:
\( a'\sb{0} | a'\sb{1} | a\sb{2} | a\sb{3} | a\sb{4} | a\sb{5} | a\sb{6} | a\sb{7} | a\sb{8} | a\sb{9} \)

\( z\sb{2} \leftarrow a\sb{2} \boxplus a\sb{5}\)
\( a'\sb{2} \leftarrow a\sb{2} \oplus a\sb{7} \)

Innerer Zustand in Runde 3:
\( a'\sb{0} | a'\sb{1} | a'\sb{2} | a\sb{3} | a\sb{4} | a\sb{5} | a\sb{6} | a\sb{7} | a\sb{8} | a\sb{9} \)

\( z\sb{3} \leftarrow a\sb{3} \boxplus a\sb{6}\)
\( a'\sb{3} \leftarrow a\sb{3} \oplus a\sb{8} \)

Innerer Zustand in Runde 4:
\( a'\sb{0} | a'\sb{1} | a'\sb{2} | a'\sb{3} | a\sb{4} | a\sb{5} | a\sb{6} | a\sb{7} | a\sb{8} | a\sb{9} \)

\( z\sb{4} \leftarrow a\sb{4} \boxplus a\sb{7}\)
\( a'\sb{4} \leftarrow a\sb{4} \oplus a\sb{9} \)

Innerer Zustand in Runde 5:
\( a'\sb{0} | a'\sb{1} | a'\sb{2} | a'\sb{3} | a'\sb{4} | a\sb{5} | a\sb{6} | a\sb{7} | a\sb{8} | a\sb{9} \)

\( z\sb{5} \leftarrow a\sb{5} \boxplus a\sb{8}\)
\( a'\sb{5} \leftarrow a\sb{5} \oplus a'\sb{0} , (a\sb{0} \oplus a\sb{5}) \) 

Innerer Zustand in Runde 6:
\( a'\sb{0} | a'\sb{1} | a'\sb{2} | a'\sb{3} | a'\sb{4} | a'\sb{5} | a\sb{6} | a\sb{7} | a\sb{8} | a\sb{9} \)

\( z\sb{6} \leftarrow a\sb{6} \boxplus a\sb{9}\)
\( a'\sb{6} \leftarrow a\sb{6} \oplus a'\sb{1} , (a'\sb{1} \leftarrow a\sb{1} \oplus a\sb{6}) \) 

Innerer Zustand in Runde 7:
\( a'\sb{0} | a'\sb{1} | a'\sb{2} | a'\sb{3} | a'\sb{4} | a'\sb{5} | a'\sb{6} | a\sb{7} | a\sb{8} | a\sb{9} \)

\( z\sb{7} \leftarrow a\sb{7} \boxplus a'\sb{0} , (a\sb{0} \oplus a\sb{5}) \)
\( a'\sb{7} \leftarrow a\sb{7} \oplus a'\sb{2} , (a\sb{2} \oplus a\sb{7}) \) 

Innerer Zustand in Runde 8:
\( a'\sb{0} | a'\sb{1} | a'\sb{2} | a'\sb{3} | a'\sb{4} | a'\sb{5} | a'\sb{6} | a'\sb{7} | a\sb{8} | a\sb{9} \)

\( z\sb{8} \leftarrow a\sb{8} \boxplus a'\sb{1} , (a\sb{1} \oplus a\sb{6}) \)
\( a'\sb{8} \leftarrow a\sb{8} \oplus a'\sb{3} , (a\sb{3} \oplus a\sb{8}) \) 

Innerer Zustand in Runde 9:
\( a'\sb{0} | a'\sb{1} | a'\sb{2} | a'\sb{3} | a'\sb{4} | a'\sb{5} | a'\sb{6} | a'\sb{7} | a'\sb{8} | a\sb{9} \)

\( z\sb{9} \leftarrow a\sb{9} \boxplus a'\sb{2} , (a\sb{2} \oplus a\sb{7}) \)
\( a'\sb{9} \leftarrow a\sb{9} \oplus a'\sb{4} , (a\sb{4} \oplus a\sb{9}) \) 

Innerer Zustand in Runde 10:
\( a'\sb{0} | a'\sb{1} | a'\sb{2} | a'\sb{3} | a'\sb{4} | a'\sb{5} | a'\sb{6} | a'\sb{7} | a'\sb{8} | a'\sb{9} \)

\( z\sb{10} \leftarrow a'\sb{0} \boxplus a'\sb{3} , (a\sb{0} \oplus a\sb{5}) \boxplus (a\sb{3} \oplus a\sb{8}) \)
\( a''\sb{0} \leftarrow a'\sb{0} \oplus a'\sb{5} , (a\sb{0} \oplus a\sb{5}) \oplus (a\sb{5} \oplus (a\sb{0} \oplus a\sb{5})) \) 

Innerer Zustand in Runde 11:
\( a''\sb{0} | a'\sb{1} | a'\sb{2} | a'\sb{3} | a'\sb{4} | a'\sb{5} | a'\sb{6} | a'\sb{7} | a'\sb{8} | a'\sb{9} \)

\( z\sb{11} \leftarrow a'\sb{1} \boxplus a'\sb{4} , (a\sb{1} \oplus a\sb{6}) \boxplus (a\sb{4} \oplus a\sb{9}) \)
\( a''\sb{1} \leftarrow a'\sb{1} \oplus a'\sb{6} , (a\sb{1} \oplus a\sb{6}) \oplus (a\sb{6} \oplus (a\sb{1} \oplus a\sb{6})) \) 

Innerer Zustand in Runde 12:
\( a''\sb{0} | a''\sb{1} | a'\sb{2} | a'\sb{3} | a'\sb{4} | a'\sb{5} | a'\sb{6} | a'\sb{7} | a'\sb{8} | a'\sb{9} \)

\( z\sb{12} \leftarrow a'\sb{2} \boxplus a'\sb{5} , (a\sb{2} \oplus a\sb{7}) \boxplus (a\sb{5} \oplus (a\sb{0} \oplus a\sb{5})) \)
\( a''\sb{2} \leftarrow a'\sb{2} \oplus a'\sb{7} , (a\sb{2} \oplus a\sb{7}) \oplus (a\sb{7} \oplus (a\sb{2} \oplus a\sb{7})) \) 

Innerer Zustand in Runde 13:
\( a''\sb{0} | a''\sb{1} | a''\sb{2} | a'\sb{3} | a'\sb{4} | a'\sb{5} | a'\sb{6} | a'\sb{7} | a'\sb{8} | a'\sb{9} \)

\( z\sb{13} \leftarrow a'\sb{3} \boxplus a'\sb{6} , (a\sb{3} \oplus a\sb{8}) \boxplus (a\sb{6} \oplus (a\sb{1} \oplus a\sb{6})) \)
\( a''\sb{3} \leftarrow a'\sb{3} \oplus a'\sb{8} , (a\sb{3} \oplus a\sb{8}) \oplus (a\sb{8} \oplus (a\sb{3} \oplus a\sb{8})) \) 

Innerer Zustand in Runde 14:
\( a''\sb{0} | a''\sb{1} | a''\sb{2} | a''\sb{3} | a'\sb{4} | a'\sb{5} | a'\sb{6} | a'\sb{7} | a'\sb{8} | a'\sb{9} \)

\( z\sb{14} \leftarrow a'\sb{4} \boxplus a'\sb{7} , (a\sb{4} \oplus a\sb{9}) \boxplus (a\sb{7} \oplus (a\sb{2} \oplus a\sb{7})) \)
\( a''\sb{4} \leftarrow a'\sb{4} \oplus a'\sb{9} , (a\sb{4} \oplus a\sb{9}) \oplus (a\sb{9} \oplus (a\sb{4} \oplus a\sb{9})) \) 

Innerer Zustand in Runde 15:
\( a''\sb{0} | a''\sb{1} | a''\sb{2} | a''\sb{3} | a''\sb{4} | a'\sb{5} | a'\sb{6} | a'\sb{7} | a'\sb{8} | a'\sb{9} \)

\( z\sb{15} \leftarrow a'\sb{5} \boxplus a'\sb{8} , (a\sb{5} \oplus (a\sb{0} \oplus a\sb{5})) \boxplus (a\sb{8} \oplus (a\sb{3} \oplus a\sb{8})) \)
\( a''\sb{5} \leftarrow a'\sb{5} \oplus a''\sb{0} , (a\sb{5} \oplus a\sb{0}) \oplus ((a\sb{0} \oplus a\sb{5}) \oplus (a\sb{5} \oplus (a\sb{0} \oplus a\sb{5}))) \) 
\end{alltt}

\end{appendix}

\newpage


\nocite{*}
\begin{thebibliography}{15}
\bibitem{LFSR}
LFSR \nolinkurl{http://en.wikipedia.org/LFSR}

\end{thebibliography}

\end{document}
